% 如果需要更新, 請email至: wufish@gmail.com
% 也可以到github留言: https://github.com/coldwufish/NYCU-thesis-template

% 載入會用到的套件, 通常不需要修改這邊的資料
%\documentclass[12pt, draftclsnofoot, onecolumn]{IEEEtran}  % IEEE one-column 樣板
\documentclass[12pt,a4paper,oneside]{book}                  % NYCU碩論樣板
%\documentclass[conference]{IEEEtran}
%\linespread{2}

\usepackage[ top=2.5cm,bottom=2.5cm,left=3cm,right=2cm ]{ geometry }

%\documentclass[oneside,zh]{nctuthesis}
%\documentclass[conference]{IEEEtran}
%\documentclass[onecolumn, journal, 12pt]{IEEEtran}
%\documentclass[12pt,draftcls,onecolumn]{IEEEtran}
%\setlength{\baselineskip}{12.7pt}
\usepackage{subfigure}
\usepackage{caption}
\usepackage{CJKutf8}
\usepackage{indentfirst}
\usepackage{array}
\usepackage{multirow}
\usepackage{amsthm}
\usepackage{amsmath}
\usepackage{amsfonts}
\usepackage{amssymb}
\usepackage{graphicx}
\usepackage{multirow}
\usepackage{setspace}
\usepackage{cite}
\usepackage{balance}
\usepackage{textcomp}
\usepackage{color}
\usepackage[ampersand]{easylist}
\ListProperties(Hide=100, Hang=true, Progressive=3ex, Style*=$\bullet$ ,
Style2*=\tiny$\blacksquare$ ,Style3*=$\circ$ ,Style4*=-- )
\usepackage{changepage}
\usepackage{algorithm}
\usepackage{algpseudocode}
\usepackage{graphics}
\usepackage{epsfig}
\floatname{algorithm}{Algorithm}
\renewcommand{\algorithmicrequire}{\textbf{Input:}}
\renewcommand{\algorithmicensure}{\textbf{Output:}}
\renewcommand{\algorithmiccomment}{// }
\usepackage{verbatim}
\newtheorem{theorem}{Theorem}
\newtheorem{lemma}{Lemma}
\newtheorem{prop}{Proposition}
\newtheorem{defn}{Definition}

\usepackage{eqparbox}
\usepackage{textcomp}

\usepackage[T1]{fontenc}
\usepackage{enumerate}

\renewcommand*\rmdefault{ptm}
\usepackage[font={rm}]{caption}

\renewcommand\algorithmiccomment[1]{%
  \hfill\#\ \eqparbox{COMMENT}{#1}%
}
\usepackage{verbatim}

\usepackage{indentfirst}

\usepackage{chngcntr}
\counterwithout{figure}{chapter}
\counterwithout{table}{chapter}

\usepackage{color}
\usepackage{colortbl}
\newcommand{\textred}[1]{\textcolor{red}{#1}}
\newcommand{\textblue}[1]{\textcolor{blue}{#1}}
\newcommand{\textcyan}[1]{\textcolor{cyan}{#1}}
\newcommand{\myHuge}[1]{\fontsize{40}{50} #1}


% source code hightlighting
\usepackage{listings}
\lstset{
	numbers=left,
	stepnumber=1,
	firstnumber=1,
	captionpos=b,
	tabsize=2,
	basicstyle=\small,
	numberfirstline=true
}

% setting the page number to footer
\usepackage{fancyhdr}
\fancyhf{}
\cfoot{\thepage}
\pagestyle{fancy}
% no header and footer bar
\renewcommand{\headrulewidth}{0pt}
\renewcommand{\footrulewidth}{0pt}

% line height setting
\linespread{1.5}
\usepackage{setspace}

%%%%%%%%%%%%%%%%%%%%%%%%%%%%%% Textclass specific LaTeX commands.
\usepackage{pdfpages}

% 浮水印套件 + 語法
\usepackage{transparent}
\usepackage{eso-pic}
\newcommand\MyAtPageCenter[1]{\AtPageUpperLeft{%
\put(\LenToUnit{.42\paperwidth},\LenToUnit{-.5\paperheight}){#1}}%
}
% 浮水印套件 + 語法 end

\usepackage[center] {titlesec} %chapter1修改为第1章
\usepackage{zhnumber} % 轉換中文數字
\usepackage{titletoc, CJKnumb}
\usepackage{etoolbox}
\newtoggle{toc-use-cn}
\newtoggle{iamphd}

\usepackage{adjustbox}


\usepackage{xstring}
\usepackage{anyfontsize}
 

% 為了看起來比較協調, 目錄的語言改為全部英文 or 全部中文, 中英混雜有點奇怪.
% 要使用中文目錄可把下面的註解拿掉, 當\toggletrue{toc-use-cn}啟用才會使用中文目錄, 預設使用英文目錄
%\toggletrue{toc-use-cn}

% --------------------------------------------
% 在這邊寫自己的資料

% 論文名稱
\newcommand{\chineseTitle}{中文論文名字}
\newcommand{\englishTitle}{English Title}

% 自己的名字
\newcommand{\studentCnName}{學生名字}
\newcommand{\studentEnName}{XXXX Wu} % 書名頁
\newcommand{\studentEnNameA}{Wu, XXXX} % 封面用
% 英文名字有兩種寫法, 一個是姓在後, 一個是放前面. 

% 教授的名字
\newcommand{\advisorCnName}{指導教授名字}
\newcommand{\advisorEnName}{OOOO Tseng}     % 書名頁
\newcommand{\advisorEnNamex}{Tseng, OOOO}   % 封面用

% 共同指導教授的名字, 若有再填寫即可
%\newcommand{\advisorCnNameB}{共同指導教授名字}
%\newcommand{\advisorEnNameB}{OOO Lin}     % 書名頁
%\newcommand{\advisorEnNameBx}{Lin, OOO}   % 封面用


% 論文的日期
\newcommand{\ThesisDate}{August 2022}           
\newcommand{\ThesisDateTW}{中華民國~ 一一一年八月} 

% 博士班就把下方註解拿掉吧!!
%\toggletrue{iamphd}

% [書名頁] 各學院、系所、學位中文、英文名稱: 
\newcommand{\CollegeCnName}{資訊學院} % 這個其實沒用到, 不過為了一致還是放進來
\newcommand{\CollegeEnName}{College of Computer Science}
\newcommand{\DepartInstitCnName}{網路工程研究所} % 系所名稱, 請看下方[說明1]
\newcommand{\DepartInstitEnName}{Institute of Network Engineering}
\newcommand{\DegreeName}{Master of Science} % 學位名稱, 常見的是Master of Science, 不過這個有很多寫法, 要記得查學校的資料.
\newcommand{\ResearchTopic}{Computer Science} % 研究領域, 請看下方[說明2]

% [說明1]
% 這邊需要填寫 XX系 or OO所 的名稱
% 如果你的單位是系所合一, 就用 Department. 如: 土木系碩士班請用 Department
% 只有所的話, 就用Institute. 如: 電信所請用Institute


% [說明2]
% 書名頁的最後會有 in OOOO 的內容. 這個目前找不到通用性的寫法. 
% 大家就自行找前人的畢業論文, 看同系所的人是怎麼寫, 就跟他們寫一樣的吧.
% 不過有些系所不需要寫這個東西, ex: 教育所. 不需要的話就把\ResearchTopic整句刪掉

% 中英文名稱請看學校網站: https://aa.nycu.edu.tw/reg/統計資訊/
% --------------------------------------------


% --------------------------------------------
% --------------------------------------------
% 填寫範例-1
% 博士班範例
%\toggletrue{iamphd}
%\newcommand{\CollegeCnName}{電機學院} 
%\newcommand{\CollegeEnName}{College of Electrical and Computer Engineering}
%\newcommand{\InstituteCnName}{電信工程研究所}
%\newcommand{\InstituteEnName}{Institute of Communications Engineering}
%\newcommand{\DegreeName}{Doctor of Philosophy} 
%\newcommand{\ResearchTopic}{Electronics and Electrical Engineering}

% 碩士班範例
%\newcommand{\CollegeCnName}{資訊學院} 
%\newcommand{\CollegeEnName}{College of Computer Science}
%\newcommand{\InstituteCnName}{網路工程研究所}
%\newcommand{\InstituteEnName}{Institute of Network Engineering}
%\newcommand{\DegreeName}{Master of Science}
%\newcommand{\ResearchTopic}{Computer Science} 
% --------------------------------------------
% --------------------------------------------



\begin{document}
\begin{CJK*}{UTF8}{bkai}

% =============================================================
% 封面的設定, 像是日期之類的 settings for cover 
\newgeometry{top=3cm,bottom=3cm,left=3cm,right=3cm}

% 1. 第一頁的封面, 記得修改系所
\begin{titlepage}

  \begin{center}
    \vspace{0.6cm}
    \fontsize{18}{22}\selectfont{國立陽明交通大學} \\
    \fontsize{18}{22}\selectfont{\DepartInstitCnName} \\  
    \fontsize{18}{22}\selectfont{\iftoggle{iamphd}{博士論文}{碩士論文}}\\ 

    \vspace{1cm}
    
\fontsize{14}{17}\selectfont{\DepartInstitEnName} \\
\fontsize{16}{19}\selectfont{National Yang Ming Chiao Tung University} \\ 
\fontsize{16}{19}\selectfont{\iftoggle{iamphd}{Doctoral Dissertation}{Master Thesis}} \\ 

    \vspace{3.1cm}  
    \fontsize{18}{22}\selectfont \chineseTitle \\
    \fontsize{18}{22}\selectfont \englishTitle \\
    \vspace{3.1cm}

    \begin{tabular}{@{\hspace{1em}} r l}
    研\ \ 究\ \ 生: & {\studentCnName}  (\studentEnNameA)  \\
    指導教授: & {\advisorCnName} (\advisorEnNamex)  \\
    \ifdefined\advisorCnNameB % 如果有共同指導教授
      & {\advisorCnNameB} (\advisorEnNameBx)
    \fi
   
    \end{tabular}
    
    \vspace{\fill}
    
    \fontsize{18}{22}\selectfont \ThesisDateTW \\
    \fontsize{18}{22}\selectfont \ThesisDate
    
  \end{center}

\end{titlepage}        % 論文的封面

% 2. 第二頁的書名頁, 記得修改系所, 日期
% 目前的浮水印剛好可以把校名&相關資訊包在裡面
% 如果論文名稱太長的話, 換行之後的外觀就沒那麼漂亮XD 有需要的話可以自行調成inside裡面的字體大小. 
% 目前是\LARGE, 可以再小一點: \Large, 再小一點: \large
% 書名頁起至最後一頁皆須加入浮水印
\AddToShipoutPicture*{
    \put(-30,0){
        \parbox[b][\paperheight]{\paperwidth}{%
            \vfill
            \centering
            {\transparent{0.2}\includegraphics[scale=0.6]{covers/logo.png}}%
            \vfill
        }
    }
}

\begin{titlepage}
  \begin{center}

    \begin{minipage}[t][3cm][t]{\textwidth}
  
    % ======================================
    % 使用變數決定標題文字大小, 這邊的寫法是用一列來呈現所有文字
    % 如果覺得執行結果很奇怪, 可以把這一段註解掉, 改用另一種呈現方式
    \ifthenelse{\chineseTitleLength > \chinesethresh}
    {
        \begin{adjustbox}{width=0.95\linewidth}
        \chineseTitle
        \end{adjustbox}
        \vspace{-0.4cm}
    }
    { % else
        \centering \LARGE \chineseTitle 
    }

    \ifthenelse{\englishTitleLength > \englishthresh}
    {
        \begin{adjustbox}{width=0.96\linewidth}
        \englishTitle
        \end{adjustbox}
        \vspace{-0.4cm}
    }
    { % else
        \centering \LARGE \englishTitle 
    }

    % 使用變數決定標題文字大小 end
    % ======================================

    % ======================================
    % 另一種呈現方式
    % 預設是\LARGE, 也可以小一點: \Large, 再小一點: \large
    % \LARGE \chineseTitle \\
    % \LARGE \englishTitle \\
    % ======================================
  
    %\vspace{1.3cm}

    \end{minipage}
  
    \fontsize{14}{14}\selectfont{
    \begin{tabular}{r l c r l}
    研\ \ 究\ \ 生: & \studentCnName & \hspace{0.6cm} & Student: & \studentEnName \\
    指導教授: & \advisorCnName ~ 博士 & \hspace{0.6cm} & Advisor: & Dr. \advisorEnName \\
    \ifdefined\advisorCnNameB % 如果有共同指導教授
      & \advisorCnNameB ~ 博士 & & & Dr. \advisorEnNameB
    \fi
    \end{tabular}
    }
    
    \vspace{1.3cm}

    \fontsize{14}{17}\selectfont{國立陽明交通大學} ~\\
    \fontsize{14}{17}\selectfont{\DepartInstitCnName} ~\\  
    \fontsize{14}{17}\selectfont{\iftoggle{iamphd}{博士論文}{碩士論文}} ~\\     



    \vspace{1.8cm}

    \fontsize{14}{14}\selectfont{
    A \iftoggle{iamphd}{Dissertation}{Thesis} ~\\
    Submitted to \DepartInstitEnName \\
    \CollegeEnName \\
    National Yang Ming Chiao Tung University \\
    in partial Fulfillment of the Requirements \\
    for the Degree of \\
    \DegreeName \\
    \ifdefined\ResearchTopic
      in \\ \ResearchTopic
    \fi
    }
    
    \vspace{\fill}

    \ThesisDate \\
    Taiwan, Republic of China \\
    ~ \\

    \fontsize{18}{22}\selectfont  \ThesisDateTW


  \end{center}
\end{titlepage}

       % 論文的書名頁


\restoregeometry
% =============================================================

% 書名頁起至最後一頁皆須加入浮水印
\AddToShipoutPicture{
    \put(-30,0){
        \parbox[b][\paperheight]{\paperwidth}{%
            \vfill
            \centering
            {\transparent{0.2}\includegraphics[scale=0.6]{covers/logo.png}}%
            \vfill
        }
    }
}

% =============================================================

% 口試結束後, 會有一些文件(3&5)需要口委們簽名
% 這邊的東西是最後上傳到圖書館要加入的東西
% (我當年畢業不需要4 XD)

% 3. 論文電子檔著作權授權書: auth.pdf
%\includepdf[pages={1},pagecommand={\thispagestyle{empty}}]{auth.pdf}

% 4. 博士論文指導教授推薦書(碩士論文免附): phd_recommend.pdf
%\includepdf[pages={1},pagecommand={\thispagestyle{empty}}]{phd_recommend.pdf}

% 5. 學位論文審定同意書(審定書): approval_ch.pdf
%\includepdf[pages={1},pagecommand={\thispagestyle{empty}}]{approval_ch.pdf}

% ps. 圖書館有說: 授權書&審定書不用上傳,但要裝訂於紙本論文中。

% =============================================================
% 目錄設定

\frontmatter
\pagenumbering{roman}
{\fontfamily{ptm}\selectfont

\iftoggle{toc-use-cn}
{ % true section. 使用中文
\renewcommand{\contentsname}{目錄} % 使用中文目錄

% 下面這些是要在目錄上加入...的符號與頁碼
\titlecontents{chapter}[0em]{}{第\CJKnumber{\thecontentslabel}章 \hspace{0.5em}}{}{\titlerule*{.}\contentspage}[\addvspace{1em}]
\titlecontents{section}[1.5em]{\addvspace{-0.5em}}{\thecontentslabel \hspace{1em}}{}{\titlerule*{.}\contentspage}[\addvspace{0.5em}]
\titlecontents{subsection}[3em]{}{\thecontentslabel \hspace{1em}}{}{\titlerule*{.}\contentspage}[\addvspace{0.5em}]

% 6. 致謝 Acknowledgement
\addcontentsline{toc}{chapter}{誌\,\,\,\,\,謝} 
% --- Acknowledgment ---
\begin{CJK*}{UTF8}{bkai}
\begin{center}
\Large
\textbf{誌~~~~~~謝}
\end{center}

\vspace{1cm}
\linespread{2}%
\selectfont
\hspace{0.25cm}

謝天謝地

\vspace{3cm}
\begin{flushright}
XXXXX 於

國立交通大學網路工程研究所碩士班

中華民國 \, 108 年 \, 8 \,月
\end{flushright}
\end{CJK*}
\newpage

% 7 中文摘要 chinese abstract
\addcontentsline{toc}{chapter}{中文摘要} \begin{center}
    \large
    \begin{singlespace}    
        \textbf{\chineseTitle{}} \\[0.5cm]
    \end{singlespace}

    \begin{singlespace}    
    學生:\studentCnName  \hspace{2.5cm}  指導教授:\advisorCnName \hspace{0.1cm} 博士 \\

    \ifdefined\advisorCnNameB % 如果有共同指導教授
        \hspace{9.6cm}  \advisorCnNameB ~ 博士 \\
    \fi
    \end{singlespace}

    \vspace{0.5cm}

    國立陽明交通大學\ \NameofDepartmentInstituteCN\ \iftoggle{iamphd}{博士班}{碩士班} \\[0.5cm]
    \textbf{摘\hspace{1cm}要} \\[0.5cm]

\end{center}

\normalsize 

中文摘要就從這邊開始寫.

\vspace{1cm}

% 中文摘要及關鍵詞 5-7 個 
\textbf{關鍵字:}中文, 摘要, 關鍵詞, 5-7個, 不要多, 也不要少
 \newpage

% 8. 英文摘要
\addcontentsline{toc}{chapter}{英文摘要} \begin{center}
    \large
    
    \begin{singlespace}
        \textbf{\englishTitle{}} \\[0.5cm]
    \end{singlespace}
    
    \begin{singlespace}
        Student : \studentEnName{}  \hspace{1.0cm} 
        % 兩個指導教授要寫 Advisors
        \ifdefined\advisorCnNameB
            Advisors: Dr.\, \advisorEnName \\
            \hspace{6.6cm} Dr.\, \advisorEnNameB  \\
        \else
            Advisor: Dr.\, \advisorEnName \\
        \fi
    \end{singlespace}
    
    \vspace{0.5cm}
    \begin{singlespace}
        \NameofDepartmentInstituteEN\\
        National Yang Ming Chiao Tung University\\[0.2cm]
    \end{singlespace}
    
    \textbf{Abstract} \\[0.5cm]

\end{center}

\normalsize 
  
Write your abstract here. Through computer vision technologies, ...


\vspace{1cm}

% 5-7 Keywords (English) 
\textbf{Keywords: English, keywords, five to seven, computer vision, IoT.} 
 \newpage

% 9. 目錄 中文版
\addcontentsline{toc}{chapter}{目錄} \tableofcontents \newpage

% 10. 圖片目錄 中文版
\renewcommand{\figurename}{圖} % 把caption的Figure改成"圖"
\renewcommand{\listfigurename}{圖目錄}
\renewcommand{\numberline}[1]{圖~#1\hspace*{1em}}
\addcontentsline{toc}{chapter}{圖目錄\vspace{0em}} \listoffigures \newpage

% 11. 表格目錄 中文版, 有需要再打開
\renewcommand{\tablename}{表} % 把caption的Table改成"表"
\renewcommand{\listtablename}{表目錄}
\renewcommand{\numberline}[1]{表~#1\hspace*{1em}}
\addcontentsline{toc}{chapter}{表目錄\vspace{0em}} \listoftables \newpage

% 把Chapter改成 第X章
\titleformat{\chapter}{\normalfont\huge\bfseries}{第\zhnum{chapter}章、}{0em}{}
\titleformat{\section}{\normalfont\Large\bfseries}{\thesection}{1em}{}
\titleformat{\subsection}{\normalfont\large\bfseries}{\thesubsection}{1em}{}
} % true end. 中文目錄設定結束
% ==========================================
% ==========================================
{ % false section. 使用英文目錄
\renewcommand{\contentsname}{Contents} % 使用英文目錄


% 下面這些是要在目錄上加入...的符號與頁碼
\titlecontents{chapter}[0em]{}{\thecontentslabel \hspace{1em}}{}{\titlerule*{.}\contentspage}[\addvspace{1em}]
\titlecontents{section}[1.5em]{\addvspace{-0.5em}}{\thecontentslabel \hspace{1em}}{}{\titlerule*{.}\contentspage}[\addvspace{0.5em}]
\titlecontents{subsection}[3em]{}{\thecontentslabel \hspace{1em}}{}{\titlerule*{.}\contentspage}[\addvspace{0.5em}]

% 6. 致謝 Acknowledgement
\addcontentsline{toc}{chapter}{Acknowledgement} 
% --- Acknowledgment ---
\begin{CJK*}{UTF8}{bkai}
\begin{center}
\Large
\textbf{誌~~~~~~謝}
\end{center}

\vspace{1cm}
\linespread{2}%
\selectfont
\hspace{0.25cm}

謝天謝地

\vspace{3cm}
\begin{flushright}
XXXXX 於

國立交通大學網路工程研究所碩士班

中華民國 \, 108 年 \, 8 \,月
\end{flushright}
\end{CJK*}
\newpage

% 7 中文摘要 chinese abstract
\addcontentsline{toc}{chapter}{Chinese Abstract} \begin{center}
    \large
    \begin{singlespace}    
        \textbf{\chineseTitle{}} \\[0.5cm]
    \end{singlespace}

    \begin{singlespace}    
    學生:\studentCnName  \hspace{2.5cm}  指導教授:\advisorCnName \hspace{0.1cm} 博士 \\

    \ifdefined\advisorCnNameB % 如果有共同指導教授
        \hspace{9.6cm}  \advisorCnNameB ~ 博士 \\
    \fi
    \end{singlespace}

    \vspace{0.5cm}

    國立陽明交通大學\ \NameofDepartmentInstituteCN\ \iftoggle{iamphd}{博士班}{碩士班} \\[0.5cm]
    \textbf{摘\hspace{1cm}要} \\[0.5cm]

\end{center}

\normalsize 

中文摘要就從這邊開始寫.

\vspace{1cm}

% 中文摘要及關鍵詞 5-7 個 
\textbf{關鍵字:}中文, 摘要, 關鍵詞, 5-7個, 不要多, 也不要少
 \newpage

% 8. english abstract
\addcontentsline{toc}{chapter}{English Abstract} \begin{center}
    \large
    
    \begin{singlespace}
        \textbf{\englishTitle{}} \\[0.5cm]
    \end{singlespace}
    
    \begin{singlespace}
        Student : \studentEnName{}  \hspace{1.0cm} 
        % 兩個指導教授要寫 Advisors
        \ifdefined\advisorCnNameB
            Advisors: Dr.\, \advisorEnName \\
            \hspace{6.6cm} Dr.\, \advisorEnNameB  \\
        \else
            Advisor: Dr.\, \advisorEnName \\
        \fi
    \end{singlespace}
    
    \vspace{0.5cm}
    \begin{singlespace}
        \NameofDepartmentInstituteEN\\
        National Yang Ming Chiao Tung University\\[0.2cm]
    \end{singlespace}
    
    \textbf{Abstract} \\[0.5cm]

\end{center}

\normalsize 
  
Write your abstract here. Through computer vision technologies, ...


\vspace{1cm}

% 5-7 Keywords (English) 
\textbf{Keywords: English, keywords, five to seven, computer vision, IoT.} 
 \newpage

% 9. 目錄 English version
\addcontentsline{toc}{chapter}{Contents} \tableofcontents \newpage

% 10. 圖片目錄 English version
\renewcommand{\numberline}[1]{Figure~#1\hspace*{1em}}
\addcontentsline{toc}{chapter}{List of Figures} \listoffigures \newpage

% 11. 表格目錄 English version, 有需要再打開
\renewcommand{\numberline}[1]{Table~#1\hspace*{1em}}
\addcontentsline{toc}{chapter}{List of Tables} \listoftables \newpage

% 調整內文的chapter, section, subection的顯示方式, 改成靠左對齊+沒有換行
\titleformat{\chapter}{\normalfont\huge\bfseries}{Chapter \thechapter.}{1em}{}
\titleformat{\section}{\normalfont\Large\bfseries}{\thesection}{1em}{}
\titleformat{\subsection}{\normalfont\large\bfseries}{\thesubsection}{1em}{}
} % false end. 使用英文


\mainmatter
\pagenumbering{arabic} % enabling page numbering


% =========================================================================
% 12. 論文正文, 可以每個章節一個.tex檔案 (put your statements in the following)

\chapter{Introduction}
\label{ch:intro}

語法大幅度修改(!?), 改成xelatex去編譯. 現在中文內容可以直接使用\textbf{粗體}跟\textit{斜體}了. 

還有研究一下overleaf支援的字體清單: https://bit.ly/3MocQG3

目前選擇TW-Kai, 這個字體同時支援繁體與簡體中文, 有一些特殊字可以直接顯示, 像是之前有人問過的核苷酸.

Video-based surveillance systems have been widely used in places such as plaza, office, factory, hotel, and conference hall for security purposes\cite{collins2000system},\cite{wang2013intelligent}. 

The rest of this paper is organized as follows. Chapter 2 reviews some related work. Chapter 3 introduces our system architecture. Chapter 4 explains the details of our pairing algorithm. Performance evaluation results are in Chapter 5. Conclusions are in Chapter 6.
 \newpage
\chapter{Related Work}
\label{ch:relatedwork}
This is related work. The PID issue has been widely studied in the field of computer vision and IoT by using various devices. In the field of computer vision, camera is the most popular device. Face recognition technologies are surveyed in \cite{zhao2003face}. Reference \cite{parkhi2015deep} focuses on how to collect a very large training dataset and build a very deep CNN model for face recognition, but training process is extremely computationally expensive. A hybrid RFID and computer vision system for localization and tracking of RFID tags is proposed in \cite{goller2014fusing}. Reference \cite{isasi2010location} presents a solution which combines RFID with object tracking through cameras. Reference \cite{germa2010vision} presents a fusion system consisting of an RFID reader and a camera crew on a mobile robot platform to track people. These works \cite{goller2014fusing},\cite{isasi2010location},\cite{germa2010vision} fuse data from camera and RFID, but their accuracy highly depends on the density of RFID antennas. Thus, they are not suitable for longer range PID. Reference \cite{munaro2014fast} proposes a fast multi-people tracking algorithm for service robots through RGB-D camera. In \cite{spinello2011people}, people detection is realized by dense depth data, called Histogram of Oriented Depths (HOD).  \newpage
\chapter{System Model}
\label{ch:architecture}

Our goal is to .... \newpage
\chapter{Data Fusion Algorithm}
\label{ch:method}

\begin{figure}[htb]
	\centering
	\includegraphics[width=0.6\textwidth]{img/data_preprocessing.png}
	\caption{An example for putting figure.}
	\label{fig:model}
	\vspace{-10mm}
\end{figure}

\section{Data Preprocessing}
\label{sec:SMA}
An example for section. Both LiDAR and wearable sensors will generate continuous data to our fusion server. Fig~\ref{fig:model} shows how these raw data are preprocessed. 


\subsection{2D LiDAR Data}
An example for subection. The LiDAR data preprocessing contains two parts: (i) background removal and (ii) clustering. Background removal 
....

 \newpage
\chapter{Performance Evaluation}
\label{ch:evaluation}

\vspace{-5mm}

In this section, ....
 \newpage
\chapter{Conclusions}
\label{ch:conclusion}
Write your conclusion here.
 
 \newpage
% =========================================================================


% =========================================================================
% 參考文獻的檔案 Reference file: ref.bib
\ClearShipoutPicture % 把ref的浮水印關掉

\iftoggle{toc-use-cn} % 使用中文
{\addcontentsline{toc}{chapter}{參考文獻}
\renewcommand{\bibname}{參考文獻} } 
{\addcontentsline{toc}{chapter}{References}
\renewcommand{\bibname}{References}
} 

\bibliographystyle{IEEEtran}
\bibliography{ref}
}
% =========================================================================

% NYCU參考資料:
% 臺灣博碩士論文知識加值系統 https://ndltd.ncl.edu.tw/
% 學位論文編寫事項 https://www.lib.nycu.edu.tw/custom_label?menu=64&lid=6
% 各學院、系所學位中英文名稱 https://aa.nycu.edu.tw/reg/統計資訊/
% 圖書館FB: https://www.facebook.com/NYCULIB



\end{CJK*}
\balance

\end{document}
